\documentclass[11pt,letterpaper,rotate]{article}

\pagestyle{plain}

\usepackage{graphicx}
\usepackage{placeins}
\usepackage{amsmath}

% remove the next section in the final version
\usepackage[paperwidth=275.9mm, paperheight=279.4mm]{geometry} % regular letter size is 215.9 wide by 279.44 long
\setlength{\evensidemargin}{95mm}
\usepackage[colorinlistoftodos, textwidth=65mm, shadow]{todonotes}

\newcounter{todocounter}
\newcommand{\todonum}[2][]{\stepcounter{todocounter}\todo[#1]{\thetodocounter: #2}}


\title{Vision and Scope Document \\
for \\
APO Theta Upsilon Chapter Website}
\date{September 19, 2012}

\begin{document}

\maketitle
\newpage

\tableofcontents
\listoffigures

\newpage

\section{Business Requirements}

\todo{Fill this in}

\subsection{Background, Business Opportunities, Customer Needs}

	-\todo{Proofread and spellcheck}
	-
	-Alpha Phi Omega is a national service fraternity. The fraternity is co-ed and the major focus is on providing service to the country, the community, the campus and the chapter. APO chapters also include a number of auxilliary activities for members such as get togethers for members known within the chapter as fellowship events. Most chapters keep track of data such as service hours completed by each memeber via paper documents. However, as chapter sizes grow the paper method becomes unwiedly. Case's Theta Upsilon chapter currently has 91 active members and a total of over 145 people with some involvement. There is clearly a need for a robust and easy to use system to manage not only service hour tracking but other aspects of the chapter's administration.
	-
	-Case's Theta Upsilon chapter currently has a website located at http://apo.case.edu. However, this website is currently only good for tracking service hours and the progress of a member towards completion of their requirements, known as a contract. This tracking functionality is very rudimentary. In addition the code running behind the website has been developed using a build-and-fix model leading to a mix of coding styles, no documentation, difficult to track down bugs and a very difficult codebase to add new features too. As the chapter has grown it has become desirable to add more administrative features as well as social features for the brothers to connect with. Adding these features on top of the current code base would be difficult and lead to unstable code. This leads us to believe there is an opportunity to make a new website. One that is documented, provides more administrative tools, is designed with future improvements in mind and is overall easier for all parties involved to use.


\subsection{Business Objectives and Success Criteria}

\todo{Fill this in}

\subsection{Business Risks}

\todo{Fill this in}

\section{Vision of the Solution}

	-\subsection{Vision Statement}
	-
	-For general members of the chapter the website will be a place to report what type of community service they have done and how many hours they have completed. In addition the members of the chapter will be able to check in on the status of the chapter including upcoming events and important deadlines. The website will also act as a way for brothers to find out more about one another and get in contact with each other for bonding outside of chapter events.
	-
	-For the members in charge of the chapter's administration, the website will also serve as a tool for accomplishing their jobs. These jobs include things such as budgeting, service hour approval, fellowship planning, and contract review. The tools will be well documented so that members who have not previously held a position will be able to quickly get up to speed.
	-For those outside of the chapter the website will serve as a source of information about both Case's chapter and the national fraternity. It will provide specific information about how to join and why one would want to join. This will help in recruiting new members.


\subsection{Vision Statement}

\todo{Fill this in}

\subsection{Major Features}

\todo{Fill this in}

\subsection{Assumptions and Dependencies}

\todo{Fill this in}

\section{Scope and Limitations}

\todo{Fill this in}

\subsection{Scope of Initial and Subsequent Releases}

\todo{Fill this in}

\subsection{Limitations and Exclusions}

\todo{Fill this in}

\section{Business Contexts}

\todo{Fill this in}

\subsection{Stakeholder Profiles}

\todo{Fill this in}
\begin{table}
    \begin{tabular}{|l|l|l|l|l|}
        \hline
        Stakeholder        & Major Value                                                                               & Attitude                                                                   & Major Interests                                                           & Constraints                                      \\ \hline
        Exec               & Improved ease of updating website, better communication with the entire chapter           & Very committed to the fraternity, enthusiastic to help in any way possible & Brother retention, attracting new brothers and pledges, organizing events & Website must be easy to understand how to update \\ \hline
        Brothers           & Improved communication and access to events, more organized and accessible event sign ups & Most will be willing to help, looking forward to a more modern website     & Doing service events, PR events, fellowship events, signing contracts     & None identified                                  \\ \hline
        Pledges            & Better representation of the fraternity, better access to events                          & Will enjoy the ease of use.                                                & Doing service events, PR events, fellowship events, signing contracts     & None identified                                  \\ \hline
        Interested Parties & Better representation of the fraternity                                                   & Possibly the first impression of the fraternity                            & Finding information about the fraternity                                  & None identified                                  \\
        \hline
    \end{tabular}
\end{table}

\subsection{Project Priorities}

\todo{Fill this in}
\begin{table}
    \begin{tabular}{|l|l|l|l|}
        \hline
        Dimension & Driver                   & Constraint                                                                            & Degree of Freedom                    \\ \hline
        Schedule  & ~                        & ~                                                                                     & Final iteration on December 7th 2012 \\ \hline
        Features  & ~                        & Any features scheduled must be complete and fully operational for the final iteration & ~                                    \\ \hline
        Quality   & ~                        & All security tests must pass, all exec members must be able to use without issue      & ~                                    \\ \hline
        Staff     & Two part time developers & ~                                                                                     & ~                                    \\ \hline
        Cost      & ~                        & ~                                                                                     & None projected                       \\
        \hline
    \end{tabular}
\end{table}
\end{document}
