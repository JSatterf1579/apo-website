\documentclass[11pt,letterpaper,rotate]{article}

\pagestyle{plain}

\usepackage{graphicx}
\usepackage{placeins}
\usepackage{amsmath}

% remove the next section in the final version
\usepackage[paperwidth=275.9mm, paperheight=279.4mm]{geometry} % regular letter size is 215.9 wide by 279.44 long
\setlength{\evensidemargin}{95mm}
\usepackage[colorinlistoftodos, textwidth=65mm, shadow]{todonotes}

\newcounter{todocounter}
\newcommand{\todonum}[2][]{\stepcounter{todocounter}\todo[#1]{\thetodocounter: #2}}


\title{Vision and Scope Document \\
for \\
APO Theta Upsilon Chapter Website}
\date{September 19, 2012}

\begin{document}

\maketitle
\newpage

\tableofcontents
\listoffigures

\newpage

\section{Business Requirements}


\subsection{Background, Business Opportunities, Customer Needs}

Alpha Phi Omega is a national service fraternity. The fraternity is co-ed and the major focus is on providing service to the country, the community, the campus and the chapter. APO chapters also include a number of auxilliary activities for members such as get togethers for members known within the chapter as fellowship events. Most chapters keep track of data such as service hours completed by each memeber via paper documents. However, as chapter sizes grow the paper method becomes unwiedly. Case's Theta Upsilon chapter currently has 91 active members and a total of over 145 people with some involvement. There is clearly a need for a robust and easy to use system to manage not only service hour tracking but other aspects of the chapter's administration.

Case's Theta Upsilon chapter currently has a website located at http://apo.case.edu. However, this website is currently only good for tracking service hours and the progress of a member towards completion of their requirements, known as a contract. This tracking functionality is very rudimentary. In addition the code running behind the website has been developed using a build-and-fix model leading to a mix of coding styles, no documentation, difficult to track down bugs and a very difficult codebase to add new features too. As the chapter has grown it has become desirable to add more administrative features as well as social features for the brothers to connect with. Adding these features on top of the current code base would be difficult and lead to unstable code. This leads us to believe there is an opportunity to make a new website. One that is documented, provides more administrative tools, is designed with future improvements in mind and is overall easier for all parties involved to use.

\subsection{Business Objectives and Success Criteria}

\todo{Fill this in}

\subsection{Business Risks}

\todo{Fill this in}

\section{Vision of the Solution}

\todo{Fill this in}

\subsection{Vision Statement}

\todo{Fill this in}

\subsection{Major Features}

\todo{Fill this in}

\subsection{Assumptions and Dependencies}

\todo{Fill this in}

\section{Scope and Limitations}

\todo{Fill this in}

\subsection{Scope of Initial and Subsequent Releases}

\todo{Fill this in}

\subsection{Limitations and Exclusions}

\todo{Fill this in}

\section{Business Contexts}

\todo{Fill this in}

\subsection{Stakeholder Profiles}

\todo{Fill this in}

\subsection{Project Priorities}

\todo{Fill this in}

\end{document}