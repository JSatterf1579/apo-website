\documentclass{article}

\usepackage{graphicx}
\usepackage{placeins}
\usepackage{fancyhdr}
\usepackage{longtable}
\usepackage{lastpage}

% clear the default headers
\fancyhead{}
\fancyfoot{}

\renewcommand{\headrulewidth}{0pt}
\renewcommand{\footrulewidth}{0pt}

\newcommand{\projName}{APO Chapter Organization Website}
\newcommand{\docName}{Software Requirements Document}
\newcommand{\docVersion}{Version 1.0}
\newcommand{\docDate}{October 1, 2012}
\newcommand{\req}[1]{REQ-{#1}}

\setlength\LTleft{0pt}
\setlength\LTright{0pt}



\fancypagestyle{plain}{

\headheight=10mm
\fancyhead{\begin{longtable}{@{\extracolsep{\fill}}|c|c|}
\hline
\projName & \docVersion \\ \hline
\docName & \docDate \\ \hline
\end{longtable}}

 \fancyfoot[C]{\footnotesize Page \thepage\ of \pageref{LastPage}}
}

\pagestyle{plain}

\setcounter{tocdepth}{2}


\title{\projName \\ \vspace{10 mm}
\docName \\\vspace{10 mm}
Devin Schwab\\
Jon Chan}

\date{\docDate}

\begin{document}

\maketitle

\newpage

\tableofcontents
\listoffigures

\newpage

\section{Introduction}
\subsection{Purpose}

\indent\indent This document contains the software requirements for the \projName. This website
will be used by Case's Theta Upsilon chapter of the national service fraternity Alpha Phi Omega (APO).
The website will be accessible via the internet meaning people who are not members will be able to access
the website. In this case the website will simply act as a group of static pages providing basic information.
The actual functionality of the website will be for general members (known as brothers), people currently joining (known as pledges) and members that are elected and appointed to run the chapter (known as exec members).

The requirements in the this document are designed to produce a system that will increase brother interest
in the chapter as well as to decrease the amount of time spent administering the chapter. Additional benefits will
potentially include more transparency and accountability between elected members and general members.

The chapter currently has a website located at http://apo.case.edu, however, based on a survey of members within
the chapter there have been numerous complaints and feature requests. The previous website was built many years ago
and has been maintained by many different people over the years. The updates and fixes to the code base have varied in
quality. This has lead to a difficult to maintain code base. For this reason a new website is being designed with the new
requested features in mind. Many of the requirements in this document have been derived either from the functionality
of the previous website or from feature requests directly from members of the chapter.

\subsection{Scope}

This requirements document deals only with requirements for the \projName. The \projName \,may interface with
other outside pieces of software. In this case the requirements attempt to specify the functionality the outside service has
while remaining general so that outside services can be switched out.

This document is aimed at developers, however, the use cases and project descriptions do provide a high level
description of the project.

\subsection{Document Conventions}

In this document requirements start are labeled in the following
format REQ-\#  where ``\#'' is a number.
The number in the requirement may be followed by a ``.'' and 
another number giving the requirement the form
REQ-\#.\#  This form indicates that the requirement is a
subrequirement of the first number in the requirement. 
Requirements can include any number of numbers and periods.

For example requirement 1 would be labeled \req{1} \,. The first
requirement derived from this requirement it would be labeled\req{1.1}
\,. The first requirement derived from \req{1.1} would be labeled
\req{1.1.1} \,. The number of numbers reflect how deep the requirement
is and the ordering reflects where a requirement was derived from.

Features are referred to as FE-\# where ``\#'' is a number.

\subsection{Definitions}

% Add definition to the table
\begin{longtable}{lp{8cm}}

{\bf Brother} & A general member of the chapter that has gone through
the pledging process. \\ \\
{\bf Contract} & The requirements that a member must satisfy to remain
in good standing with the chapter \\ \\
{\bf Exec Member} & A general member of the chapter (brother) that is
elected or appointed to run a specific aspect of the chapter. \\ \\
{\bf Membership Review} & Event at the end of the semester in which
the completion of each member's contract is reviewed by the chapter\\ \\
{\bf Pledge} & A member who has completed the initiation ceremony, but
not the induction ceremony. These members do not have the full
privilege of a brother\\ \\
{\bf Rushee} & A person who is interested in joining APO, but is not
yet a pledge \\ \\
{\bf Service Report} & Request for approval of community service hours
\\ \\

\end{longtable}

\section{Overall Description}

\subsection{Project Features}

\subsection{User Classes and Characteristics}

This project has 3 main classes of users:

\begin{enumerate}
  \item Exec Members
  \item Brothers and Pledges
  \item Rushee
\end{enumerate}

Each of these user classes is described in more detail in the next sections

\subsubsection{Exec Members}

These users have been elected or appointed by the chapter to run some
part of the chapter. All Exec members must also be a Brother. Exec
members have access the administrative tools of the \projName \,. One
of the goals of this project is to decrease the amount of time that
exec members spend administering the chapter.

\subsubsection{Brothers and Pledges}

This user class can be split into two subclasses. Brothers have been
inducted into the chapter, whereas pledges have not. In the actual
workings of the chapter brothers have more rights than pledges, such
as voting. However, from the perspective of this project the only
difference between brothers and pledges is the information displayed
by each of the features.

\subsubsection{Rushee}

This user class consists of potential members. These users will only
have access to informational features.

\subsection{Operating Environment}

\subsubsection{Website Operating Environment}

The website shall be capable of running on any platform which provides
an HTTP webserver. The webserver must also provide a way for scripting
responses to web requests in order to make the website respond to user
input and previously stored information.

\subsubsection{User Operating Environment}

The user shall be capable of accessing the website from any modern
browser. This includes mobile devices.

\subsection{Assumptions and Dependencies}

The website depends on a working webserver and a working internet
connection.

\section{System Features and Associated Requirements}

\subsection{FE-1: Create, update, track, and report service events and
  hours}

\subsubsection{Use case Diagram}

\subsubsection{Requirements}

\subsection{FE-2: Create, update, sign and track progress of member
  contracts}

\subsubsection{Use case Diagram}

\subsubsection{Requirements}

\subsection{FE-3: Create, update and view profiles of members and
  alumni}

\subsubsection{Use case Diagram}

\subsubsection{Requirements}

\subsection{FE-4: Update and view chapter calendar}

\subsubsection{Use case Diagram}

\subsubsection{Requirements}

\subsection{FE-5: Write and comment on chapter blog}

\subsubsection{Use case Diagram}

\subsubsection{Requirements}

\subsection{FE-6: Submit and view photos of chapter members and
  chapter events}

\subsubsection{Use case Diagram}

\subsubsection{Requirements}

\subsection{FE-7: Track and report chapter statistics}

\subsubsection{Use case Diagram}

\subsubsection{Requirements}

\section{Additional Requirements}

\subsection{Performance}

\subsection{Security}

\subsection{Software Quality Requirements}

\end{document}